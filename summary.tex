
\thispagestyle{empty}
\subsection*{The general context}

As mathematical proof complexity continues to rise, the need for automated proof
checking becomes more and more pressing. Furthermore, more and more critical
systems integrate non-trivial software, leading a need to formally prove the
absence of bugs in that software. This has led to the development of many proof
assistants. Since these software have been developed by different teams, using
different theories, and often with subtly different goal in mind, translating a
proof from one assistant to the other must be done manually, and is often non
trivial work.

This is a problem on two levels. First, since proof assistants are developed in
standard programming languages, their correctness has not been verified. This
means we need to trust the software when it tells us a proof is correct. While
most of them only have a simple \emph{core} that need to be trusted, and not the
full codebase, it would still be a further guarantee if one could translate
proof from one assistant to the other, and check its correctness in all proof
assistants, thus reducing the risk that a bug in one of them would result in an
undetected false proof. And second, manually proving basic math results in all
proof assistants is a bit of a waste, when one could ideally just take the
proofs from other proof assistants, instead of having a fragmented ecosystem, as
is the situation right now.

This has led to the development of what are called \emph{logical framework}.
These are systems that aim to be general enough so that the type systems of most
proof assistants can be encoded into them. They thus provide a common ground to
manipulate proofs from many different systems, and write translation from a
system to any other, and eventually then export the proof to the other system.
Even if it was not developed with this aim,
\emph{Isabelle}\cite{nipkow_isabellehol_2002} is a kind of logical framework.
Another such system is \emph{NuPrl}\cite{allen_nuprl_2000}, and more recent
efforts include \emph{OpenTheory}\cite{hurd_opentheory_2009}.

The one that interest us here is \emph{Dedukti}\cite{assaf_dedukti_2016}. It is
a logical framework based on the $\lambda\Pi$ calculus modulo (also written
$\lp$), with a growing support for import and export from/to many proof
assistants, including \emph{Coq}, \emph{Matita} and \emph{OpenTheory}. Its new
version is called \emph{LambdaPi}, as a reference to the underlying logic, and
since this is the version that was used in this internship, this is how it will
be refered to it in this paper.

\subsection*{The research problem}

What will interest us here is the embedding of a specific kind a type theory in
\emph{LambdaPi}. But to understand the challenge of our embedding, we need to
explain the difference between a \emph{deep} embedding and a \emph{shallow}
embedding. When encoding a type theory or a logic in another type theory, there
are two approaches one can take.  The first one, which always work, is to create
an inductive type describing the syntax of terms, types and typing derivation of
the initial logic. On can (almost) always do that, and it result in what is
called a \emph{deep} embedding, since the logic is embedded into terms of a type
inside the target logic.  The advantage of this approach is that it is very
easy, since all one need is the description of the source logic. But it results
in something unwieldy to manipulate and work with, and also needs a lot of
uninteresting work to get right (handling $\alpha$-equivalence\dots).

The other approach, that one can do when the two logic are close enough or when
the target logic is more powerful than the starting one, is to find an injection
of the terms of the initial logic into those of the target logic such that the
injection of a type is inhabited in the target logic if and only if the initial
type is inhabited in the source logic.  The big interest of this approach is
that one can use all the tooling of the target logic to work on the embedded
term. In the case of embedding in a proof assistant, that means automatic
handling of context and variable binding, $\alpha$-equivalence, type inference,
eventually tactics to create terms\dots

So the problem here was to find a shallow embedding of \emph{Extensional Type
Theory} (ETT) into LambdaPi. I will go into more detail about the property of
ETT in the relevant section (section \ref{ETT}, page \pageref{ETT}), but since
it is much more expressive in a certain sense than LambdaPi, a naïve shallow
embedding is impossible.

\thispagestyle{empty}
\subsection*{Your contribution}

Thankfully, there is a paper by Winterhalter, Sozeau and
Tabareau\cite{winterhalter_eliminating_2019} that describes a way to embed in a
shallow-ish way ETT in traditional Martin-Löf Type Theory (MLTT). The latter has a
pretty standard shallow embedding in LambdaPi, so combining those two embeddings
give an interesting embedding of ETT in LambdaPi.

What I did was first implement the standard embedding of MLTT in LambdaPi
(section \ref{MLTT}), then formalise ETT as a deep embedding in LambdaPi
(section \ref{ETT-deep}), and finally implement a function that takes a
derivation of an ETT term a yields a translation of that term in MLTT following
the algorithm given by the previously mentioned paper (section
\ref{translation}).

\subsection*{Arguments supporting its validity}

One big advantage of my implementation is that it is fully typed using the whole
power of the type system of LambdaPi. So it is pretty much correct by typing. On
the other hand, LambdaPi does not check for confluence or terminaison of the
terms, but after testing it on a few examples (section \ref{results}), it seems
correct.

Furthermore, using the expressiveness of LambdaPi actually gives smaller and
more readable proofs than the Coq implementation of the paper, despite the
absence of tactics in LambdaPi.

On the other hand, it suffers from the inefficiency of LambdaPi, which is a very
powerful unification engine but using that power for computation result in
something slow that consumes a lot of memory, failing even on relatively simple
terms. Also translating directly to the shallow embedding means the created term
cannot be rewritten to simplify it. Finally, due to the way it is done, there is
no way to translate a term without also executing it. These last two problems
could be fixed by introducing an intermediary layer to the translation, but the
first one is pretty much unavoidable for a translation written directly in
LambdaPi.

\subsection*{Summary and future work}

Once the last remaining problems are fixed, such an embedding could be used to
embed \emph{NuPrl} proofs in LambdaPi, since it uses extensional type theory as
its foundation.  Furthermore, it could be used to encode any system with a
stronger equality than MLTT, by going through ETT first. One such system that is
of interest is \emph{Cubical Type Theory}.  Finally, this algorithm could be
easily expanded to translate from \emph{Homotopy Type
System}\cite{voevodsky_simple_2013} to \emph{Two-Level Type
Systems}\cite{annenkov_two-level_2019}.
